\section{Fall 2} \label{case-2}
% TODO: die ander Bedingung die dann auch erfüllt ist???
Wird die Bedingung $b-pa \leq a/2$ nicht erfüllt, dann führen wie Fall 2 aus. 
In Fall 2 gilt $qa-b < a/2$ und wir führen $\ell$ Schritte für $i=0, ..., \ell-1$ (und anschließend noch einen $(\ell +1)$-ten Schritt) aus, die immer das verdoppeln werden was anfänglich $a$ war. \\
Dazu muss wie in Fall 1 erst $a$, dann $2a$ und allgemein $2^ia$ von einer der anderen Nummern abgezogen werden. 
Anders als in Fall 1 jedoch ziehen wir den benötigten Betrag 
\begin{itemize}
    \item von der zweiten Nummer ab (anfänglich $b$), wenn $q_i = 1$ und
    \item von der dritten Nummer ab (anfänglich $c$), wenn $q_i = 0$.
\end{itemize}

Nach der Ausführung dieser $\ell$ Schritte, ist die erste Nummer $2^\ell a$, die zweite $b- \sum_{i=0}^{\ell-1} q_i2^ia = b-(q-2^\ell)a$ und die dritte Nummer $c- \sum_{i=0}^{\ell-1} (1-q^i)2^ia$. \\
Anschließend führen wir noch einen $(\ell +1)$-ten Schritt aus. Dieser macht aus $a = 2^\ell a- (b-qa+2^\ell a) = (-b+qa)$. \\
Die Runde endet und die erste Numer ist nun genau $(-b+qa) < a/2$. Sie gleicht der Bedingung, die wir geprüft haben um uns für Fall 2 zu entscheiden.