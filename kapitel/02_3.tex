\section{Fall 1} \label{case-1}
Aus den Hilfswerten prüfen wir die Bedingung $b-pa \leq a/2$. Wenn diese eintritt, dann führen wie Fall 1 aus, wenn nicht, dann gilt $qa-b < a/2$ und wir führen Fall 2 aus. \\
In Fall 1 führen wir $k+1$ Schritte für $i=0, ..., k$ aus, die immer das verdoppeln werden was anfänglich $a$ war. Dazu muss erst $a$, dann $2a$ und allgemein $2^ia$ von einer der anderen Nummern abgezogen werden. 
Wir ziehen den benötigten Betrag 
\begin{itemize}
    \item von der zweiten Nummer ab (anfänglich $b$), wenn $p_i = 1$ und
    \item von der dritten Nummer ab (anfänglich $c$), wenn $p_i = 0$.
\end{itemize}

Nach der Ausführung dieser $k+1$ Schritte, sind die zweite und die dritte Nummer $b- \sum_{i=0}^{k} p_i2^ia = b-pa \leq a/2$ bzw. $c- \sum_{i=0}^{k} (1-p_i)2^ia$. Die zweite Nummer $b-pa \leq a/2$ gleicht der Bedingung, die wir geprüft haben um uns für Fall 1 zu entscheiden.
