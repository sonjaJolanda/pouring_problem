\chapter{Obere Schranke und Beweis für die Gültigkeit des Algorithmus} 
\textit{\textbf{Theorem 1:} Die optimale Anzahl von Schritten zur Lösung einer Instanz mit einer Gesamtliteranzahl von $n$ ist begrenzt durch $N(n) \leq (log(n))^2$.} \\

Im folgenden Abschnitt beweisen wir dass unser Algorithmus nur gültige Konfigurationen erreicht und erläutern die obere Schranke des Algorithmus. Dafür schauen wir uns die Schritte an, in welchen der Algorithmus die Konfigurationen verändert. Zuerst betrachten wir dabei Fall 1 genauer. \\
Wir müssen beweisen dass sowohl $b$ als auch $c$ aus der Anfangskonfiguration nach dem Umschütten groß genug sind um niemals negativ zu werden und so alle Schritte dieser Runde als gültig zu erklären. Für $b$ können wir einfach unsere Beobachtung $b-pa>=0$ nutzen (zur Erinnerung: $p=\lfloor b/a \rfloor$). 
Für $c$ haben wir $c-\sum_{i=0}^{k} (1-p_i) 2^i a = c-\sum_{i=0}^{k-1} (1-p_i) 2^i a = c-(\sum_{i=0}^{k-1} (2^i) - \sum_{i=0}^{k-1} (p_i 2^i))a \geq c-2^k a \geq 0$, bei der die letzte Ungleichheit aus $k=\lfloor log(p)\rfloor$ und $a \leq c$ resultiert. 
Wir schließen daraus, dass diese Runde machbar ist und zu einem Tripel führt, dessen kleinste Zahl höchstens $b-pa \leq a/2$  ist, wie gefordert. Die Anzahl der Schritte dieser Runde ist genau $k+1 = \lfloor log(p)\rfloor +1\leq \lfloor log(q)\rfloor +1$. \\

Als nächstes betrachten wir auch Fall 2 etwas genauer. Auch hier müssen wir beweisen dass diese $\ell$ Schritte tatsächlich möglich sind indem wir zeigen dass $b$ und $c$ groß genug sind. Dies ist der Fall, da wir auf der einen Seite $q \leq p+1$ haben und dadurch $b-(q-2^\ell)a \geq b-(q-2^0)a \geq b-pa \geq 0$ gilt, 
und auf der anderen Seite $c - \sum_{i=0}^{\ell -1} (1-q_i)2^ia \geq c-\sum_{i=0}^{\ell -1} (2^ia) = b-(2^\ell-1)a \geq b-(q-1)a \geq b-pa \geq 0$. Der letzte Schritt ist möglich, da $2^\ell a-(b-qa+2^\ell a)=qa-b \geq 0$.
Auch diese Runde ist machbar und endet mit genau $\ell+1=\lfloor log(q)\rfloor+1$ Schritten. Die erste Nummer des Tripels ist nun genau $qa-b<a/2$. \\

Wir haben gezeigt, wie man in beiden Fällen eine gültige Runde ausführt mit max $\lfloor log(q)\rfloor +1\leq \lfloor log(n/1)\rfloor +1$ Schritten, die mit einem Tripel $(a', b', c')$ endet, dessen kleinste Nummer maximal $a'\leq a/2$ ist. \\
Wie jede Konfiguration in aufsteigender Reihenfolge erfüllt die Anfangskonfiguration die Bedingungen $a \leq n/3$. Deshalb ist sichergestellt dass wir nach maximal $log(n-1)$ Runden eine Konfiguration erreichen, dessen kleinste Nummer maximal $n/3 * 2^{1-log(n)}=2/3 \approx 0$ ist. \\
Es reicht also aus, diesen gesamten Prozess für höchstens $\lfloor log(q)\rfloor -1$ Runden zu iterieren, und am Ende erhalten wir eine endgültige Konfiguration, deren kleinste Zahl $0$ ist. Die Gesamtanzahl der Schritte über alle Runden ist höchstens $(log(n)-1)*(log(n)+1) \leq (log(n))^2$. 