\section{Berechnung der Hilfswerte r, p und q} \label{helper}
Die Hilfswerte werden folgendermaßen berechnet:
    %$r \coloneq b/a$, 
    %$p \coloneq \lfloor r \rfloor $ und
    %$q \coloneq \lceil r \rceil $.
Dabei können wir etwas festellen. Es gilt: $0 \leq b-pa < a $ und $ 0 \leq qa - b < a $. 
Das impliziert dass $min(b-pa, qa-b) \leq a/2$, da $(b-pa)+(qa-b)=(q-p)a \leq a$. 
Außerdem betrachten wir die kleinsten binären Repräsentationen von $p$ und $q$: $p_k...p_0$ und $q_\ell...q_0$. Diese werden im Verlauf von Fall 1 und Fall 2 relevant.