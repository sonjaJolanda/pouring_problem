\chapter{Untere Schranke} 
\textit{\textbf{Theorem 2:} Anzahl der Schritte zur Lösung eines Worst-Case-Falls mit einer Gesamtliterzahl von n ist mindestens $\lceil log ((n+1)/5) \rceil = \Omega (log(n))$} \\

Bisher haben wir nur die obere Schranke des Algorithmus betrachtet. In diesem Kapitel werden wir uns nun mit der unteren Schranke beschäftigen. \\

Sei $t:=n/3$ und $(a,b,c)$ je nach Rest von $mod 3$: 
\begin{equation}
    (a, b, c) = \begin{cases}
      (t-1, t, t+1) & \text{für } t-\lfloor t\rfloor = 0, \text{i.e.,} n \equiv 0 (mod (3))\\
      (t-(4/3), t-(1/3), t+(5/3)) & \text{für } t-\lfloor t\rfloor = 1/3, \text{i.e.,} n \equiv 1 (mod (3))\\
      (t-(5/3), t+(1/3), t+(4/3)) & \text{für } t-\lfloor t\rfloor = 2/3, \text{i.e.,} n \equiv 2 (mod (3))
    \end{cases}
 \end{equation}
Allgemein: $(t+d_1, t+d_2, t+d_3)$ mit $d_1<d_2<d_3$. \\

Dies können wir verallgemeinern zu einer Konfiguration nach $i$ Schritten. Nach 1 Schritt haben wir entweder $(2t+2d_1, d_2-d_1,t+d_3)$ oder $(2t+2d_1, t+d_2, d_3-d_1)$ oder $(t+d_1, 2t+2d_2,d_3-d_2)=(x_1, t+y_1, 2t+z_1)$.
Allgemein haben wir nach i Schritten: $(x_i, t+y_i, 2t+z_i)$. Daraus extrahieren wir die Werte $u$, $v$, $w$ und überlegen uns was dafür gilt. \\ 
Seien ${u_i, v_i, w_i }={|x_i |, |y_i |, |z_i |}$ und $u_i\leq v_i\leq w_i$. 
Für $i=1$ können wir direkt überprüfen dass $w_i\leq 5/3*2$ und $v_i\leq 5/3*2 -1/3$ ist. Allgemein gilt $w_i \leq 2w_{i-1}$ und $v_i\leq v_{i-1}+w_{i-1}$. Durch Induktion erhalten wir $u_i\leq v_i\leq w_i\leq 5/3*2^i$ und $v_i\leq 5/3*2^i  -1/3$ $\Rightarrow v_i+w_i\leq 5/3*2^(i+1) -1/3$. \\

Solange $v_i+w_i<t$ ist, können keine zwei der drei Zahlen die Summe $t$ ergeben, so dass $a, b, c$ nicht gleich sein können. Der einzige Weg zu einer Konfiguration, die eine $0$ enthält, führt jedoch über eine Konfiguration mit zwei gleichen Zahlen. Somit ist $v_i+w_i \geq t=n/3$ eine notwendige Bedingung für $(x_i, t+y_i, 2t+z_i)$, die Konfiguration nach Schritt $i$, die zwei gleiche Zahlen enthält. \\ 

Unsere Bedingung ist nicht erfüllt, solange diese beiden äquivalenten Ungleichungen wahr bleiben: $5/3*2^{i+1} -1/3<n/3  \Leftrightarrow i+1<log((n+1)/5)$. \\
Folglich muss jede Schrittzahl $k$, die uns eine Chance bietet, zwei gleiche Zahlen zu erhalten, $k+1\geq log(((n+1))/5)$ erfüllen. $\Rightarrow k\geq \lceil log((n+1)/5)  \rceil-1$. \\
Erst nach mindestens $k$ Schritten könnten zwei gleiche Zahlen in unserer Konfiguration auftauchen. Ein zusätzlicher Schritt mit diesen beiden Zahlen ist dann erforderlich, um eine Null zu erzeugen. Daher ist die Mindestanzahl der Schritte mindestens $\lceil log(((n+1))/5) \rceil= \Omega (log(n))$.